\section*{Introduzione}
Un dominio è un identificatore (univoco) utilizzato per accedere a risorse su Internet. 
In particolare, i domini sono stringhe alfanumeriche strutturate gerarchicamente, 
tradotte in indirizzi IP attraverso il \textbf{Domain Name System} (DNS).
\\\\ 
Ad esempio, il dominio \textit{example.com} è associato a un indirizzo IP che permette ai dispositivi di localizzare il server 
corrispondente. I domini sono progettati per facilitare la navigazione su Internet e per essere 
gestibili dagli esseri umani, a differenza degli indirizzi IP numerici. 
\\\\
I \textbf{registrar} sono gli enti che permettono la registrazione 
e la gestione dei domini, mentre i \textbf{name server} sono i server responsabili 
della risoluzione dei nomi di dominio. Concludiamo poi introducendo i \textbf{record DNS}, 
delle tuple che contengono le informazioni associate a un dominio, come gli indirizzi IP (record \verb|A| e \verb|AAAA|), 
i server mail (\verb|MX|) e i riferimenti a server di nomi (\verb|NS|).

\subsection*{Classificazione dei domini}
I domini possono essere classificati secondo diversi criteri.
\\\\
In base alla gerarchia del DNS, si dividono in: 
\begin{itemize}
    \item \textbf{top-level domains} (TLD) come \verb|.com|, \verb|.net|, \verb|.org|, \verb|.gov|
    e \verb|.edu|.
    \item \textbf{second-level domains} (SLD) come \verb|example| in \verb|example.com|.
    \item \textbf{subdomains} (o \textit{third-level domains}) utili per l'organizzazione 
    delle risorse, come \verb|blog| in \verb|blog.example.com|.
\end{itemize} 
Sulla base dell'accessibilità, si dicono invece:
\begin{itemize}
    \item \textbf{generici} (gTLD), registrabili pubblicamente, come \verb|.com|, \verb|.net| e \verb|.info|.
    \item  \textbf{nazionali} (ccTLD), assegnati a specifiche nazioni, come \verb|.it| per l'Italia o \verb|.fr| per la 
    Francia.
    \item \textbf{sponsorizzati} (sTLD), gestiti da specifiche organizzazioni (o settori), come 
    \verb|.gov| per enti governativi e \verb|.edu| per istituzioni educative.
    \item \textbf{riservati}, non registrabili pubblicamente, come \verb|.localhost| e \verb|.test|. 
\end{itemize} 
Infine, in base allo scopo per cui vengono registrati, si distinguono in:
\begin{itemize}
    \item \textbf{legittimi}, registrati e utilizzati per siti web, aziende e servizi autentici.
    \item \textbf{malevoli}, generati da malware o cybercriminali per attacchi informatici.
    \item \textbf{parcheggiati}, registrati ma non  utilizzati attivamente, spesso con l'intento di rivenderli o per 
    protezione del marchio.
\end{itemize}