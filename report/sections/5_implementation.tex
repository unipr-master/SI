\section*{Implementazione}
Di seguito è riportato un esempio, in Python, di DGA basato su una 
funzione pseudo-casuale. L'algoritmo prende come input una data (anno, 
mese e giorno) e restituisce un nome di dominio generato dinamicamente.
\\\\
L'idea alla base di questo codice è quella di creare una 
sequenza pseudo-casuale di caratteri, in base ai valori 
della data fornita.
\\\\
\begin{lstlisting}
    def generate_domain(year: int, month: int, day: int) -> str:
    
    """Generate a domain name for the given date."""
    domain = ""
    
    for i in range(16):
        year = ((year ^ 8 * year) >> 11) ^ ((year & 0xFFFFFFF0) << 17)
        month = ((month ^ 4 * month) >> 25) ^ 16 * (month & 0xFFFFFFF8)
        day = ((day ^ (day << 13)) >> 19) ^ ((day & 0xFFFFFFFE) << 12)
        domain += chr(((year ^ month ^ day) % 25) + 97)
    
    return domain + ".com"
\end{lstlisting}

In particolare, ogni iterazione del ciclo modifica i valori di \texttt{year}, 
\texttt{month} e \texttt{day} utilizzando operazioni bitwise (\texttt{shift}, 
\texttt{XOR} e \texttt{AND}) per manipolare i valori numerici e trasformarli 
in caratteri alfabetici.

Poiché il ciclo viene eseguito 16 volte, il risultato finale sarà 
una stringa di 16 caratteri alfabetici, seguiti dal suffisso \texttt{.com}.
\\\\
Ad esempio, l'esecuzione del codice, dando come input il 7 gennaio 2014, 
produce in output il dominio

\texttt{intgmxdeadnxuyla.com}
