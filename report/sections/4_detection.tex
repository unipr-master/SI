\section*{Tecniche di rilevamento}
Il rilevamento dei domini generati tramite DGA prevede diversi 
approcci

\subsection*{Analisi lessicale e sintattica}
Uno dei primi metodi utilizzati per identificare i domini generati 
algoritmicamente è l'analisi delle loro caratteristiche \textbf{lessicali} 
e \textbf{sintattiche}. Come anticipato, infatti, molti DGA, generano 
domini privi di significato, spesso caratterizzati da lunghe stringhe 
alfanumeriche casuali.
Gli algoritmi di rilevamento basati su questa tecnica esaminano quindi
parametri come la lunghezza del dominio, la frequenza delle lettere, 
la presenza di sillabe comuni, l'entropia della stringa 
e la distribuzione dei caratteri.  
\\\\
Ad esempio, un dominio generato da un DGA pseudo-casuale come 
\verb|axwscwsslmiagfah.com| appare molto diverso da un dominio legittimo 
come \verb|unipr.it|. Utilizzando metriche statistiche, 
è spesso possibile classificare un dominio come sospetto. 

\subsection*{Analisi del traffico di rete}  
Un altro approccio efficace per il rilevamento dei DGA è 
l'analisi del comportamento del traffico di rete. 
A differenza dei domini legittimi, che vengono interrogati da un 
ampio numero di utenti, i domini generati dai malware presentano spesso 
comportamenti anomali.  
\\\\
Un primo indicatore è l'elevata velocità di risoluzione dei DNS. 
Un dispositivo infetto da malware che utilizza un DGA tenterà di 
connettersi a centinaia o migliaia di domini in pochi secondi,
nella speranza di trovare un dominio C2 valido. 
Questo comportamento è diverso da quello di un utente normale,
che visita un numero relativamente basso di domini in un 
determinato periodo di tempo.  
\\\\
Inoltre, poiché i domini malevoli tendono a essere attivi 
per periodi brevi, il monitoraggio della 
\textbf{longevità dei domini} può quindi diventare un valido 
indicatore per le attività malevole

\subsection*{Approcci di machine learning} 
Con l'aumento della complessità dei DGA, le tecniche tradizionali 
di rilevamento, basate su regole euristiche, si sono dimostrate sempre 
meno efficaci. Per questo motivo, i sistemi di sicurezza più complessi
implementano modelli basati sull'apprendimento automatico per individuare pattern 
ricorrenti nei domini generati. 
\\\\
L'apprendimento automatico permette infatti di addestrare modelli 
su grandi quantità di dati, distinguendo tra domini legittimi e 
domini generati algoritmicamente. I modelli di classificazione 
più utilizzati sono basati su reti neurali, alberi decisionali e 
support vector machines. 
\\\\
Un esempio di tecnica basata sul machine learning prevede l'uso 
di reti neurali convoluzionali per analizzare la struttura dei nomi 
di dominio e identificare somiglianze con campioni noti di DGA. 
Inoltre, l'utilizzo di algoritmi di clustering permette di 
raggruppare i domini con comportamenti simili, evidenziando quelli 
che potrebbero essere generati automaticamente.  

\section*{Contromisure}
Oltre al rilevamento dei DGA, esistono diverse strategie di 
prevenzione che possono ridurre l'impatto di queste minacce 
e aiutano a proteggere le reti.

\subsection*{Sinkholing}
Una delle misure più efficaci è il sinkholing, che 
consiste nel registrare preventivamente i domini 
generati dagli algoritmi noti e reindirizzare il traffico malevolo
verso server controllati, anziché verso il server C2 del malware.
In questo modo, è possibile impededire al malware di ricevere nuovi 
comandi. 
\\\\
Questa tecnica è stata ampiamente utilizzata contro Conficker

\subsection*{Blacklist dinamica}
Un'altra strategia spesso utilizza è la realizzazione di una 
blacklist dinamica, che aggiorna periodicamente l'elenco dei domini 
sospetti, identificati tramite i metodi di rilevamento sopra descritti. 
Tuttavia, poiché i malware generano continuamente nuovi domini, 
questo approccio richiede aggiornamenti costanti per rimanere efficace.  

\subsection*{Filtraggio DNS}
Un metodo più avanzato è il DNS filtering, basato su 
intelligenza artificiale, che combina tecniche di machine 
learning e analisi del traffico per identificare in tempo
reale i domini malevoli. 
\\\\
Questi sistemi possono essere implementati a livello di firewall ,
o direttamente all'interno di sistemi DNS, per bloccare 
automaticamente le richieste verso domini sospetti.  

\subsection*{Difesa proattiva}
Infine, è fondamentale adottare strategie di difesa proattiva, 
come:
\begin{itemize}
    \item il \textbf{monitoraggio continuo} del traffico DNS, che permette 
    di identificare anomalie nel traffico e rispondere
    tempestivamente a eventuali attacchi.
    \item la \textbf{segmentazione della rete}, che impedisce ai 
    malware di propagarsi all'interno della rete.
\end{itemize}