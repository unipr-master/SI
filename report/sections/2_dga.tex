
\section*{Algoritmi di generazione}
Gli algoritmi di generazione di dominio, noti come \textbf{DGA} (Domain Generation Algorithms), 
sono strumenti automatici per la creazione di domini, sulla base di schemi predefiniti.  
\\\\
Originariamente sviluppati per l'utilizzo in applicazioni legittime, 
come la creazione di indirizzi temporanei o la gestione di servizi distribuiti, 
questi algoritmi vegono oggi largamente utilizzati dal cybercrimine per
eludere i sistemi di sicurezza e mantenere attive le comunicazioni tra 
malware e i server di comando e controllo (C2). 
\\\\
Il loro funzionamento segue, generalmente, lo schema qui riportato: un malware
infetto esegue l'algoritmo che, a intervalli regolari, genera una 
lista di possibili domini, a cui tenterà di connettersi per ricevere istruzioni. 
Il server C2, a sua volta, registrerà anticipatamente uno dei domini generati, 
permettendo così la comunicazione tra l'attaccante e il sistema compromesso. 
\\\\
Questi algoritmi sfruttano funzioni deterministiche o pseudo-casuali 
per generare (in maniera prevedibile) un insieme di domini che cambiano
nel tempo, a partire da uno specifico \textit{seed} (come una data, una chiave segreta o un valore 
ottenuto da una risorsa pubblica).Quando un dominio viene bloccato 
dalle misure di sicurezza della rete, l'algoritmo continua a generare 
nuovi indirizzi, rendendo il rilevamento e la mitigazione estremamente 
complesse.

\subsection*{Topologie di algoritmi}
Gli algoritmi di generazione dei domini possono essere classificati in diverse 
categorie, a seconda della logica impiegata per la creazione dei nomi. 
\\\\
Gli algoritmi \textbf{basati su dizionario} combinano parole predefinite per 
generare nomi di dominio che risultano piuttosto realistici, e perciò 
difficili da distinguere da quelli legittimi. Questi algoritmi offrono
quindi il vantaggio di generare domini apparentemente naturali, ma la
loro efficacia dipende dalla dimensione e dalla varietà del dizionario utilizzato. 
\\\\
Un'altra tipologia comune di DGA è rappresentata dagli algoritmi \textbf{pseudo-casuali}, 
che generano domini(spesso stringhe alfanumeriche casuali), basandosi su 
un \textit{seed} (generalmente temporale) per ottenere risultati 
riproducibili. Sebbene questi domini possano variare dinamicamente, 
la loro natura casuale li rende più facili da 
rilevare con strumenti di analisi automatica.
\\\\
Alcuni algoritmi, più sofisticati, adottano \textbf{modelli statistici}
per creare domini che imitano quelli reali. Questi metodi utilizzano 
spesso modelli Markoviani o altre tecniche di apprendimento statistico 
per produrre nomi che sembrano naturali, aumentando così la difficoltà 
di rilevamento. 
\\\\
Gli approcci più avanzati si basano invece su tecniche di 
\textbf{apprendimento automatico} e reti neurali per generare 
domini che si adattano dinamicamente ai protocolli di 
sicurezza, risultando quindi in sistemi sempre più difficili da individuare. 
Sebbene questa soluzione offra un elevato livello di
resistenza alle misure di sicurezza, la sua implementazione spesso complessa, oltre a 
richiedere maggiori risorse computazionali.