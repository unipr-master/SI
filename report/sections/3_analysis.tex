\section*{DGA nelle botnet e nei malware}
Le botnet sono reti di dispositivi compromessi, controllati in modo 
remoto tramite un'infrastruttura C2. Un elemento fondamentale per il "corretto"
successo di una botnet è la capacità di comunicare con il server C2 
senza essere intercettata o bloccata. 
Per questo motivo, i malware moderni utilizzano DGA per generare 
dinamicamente nomi di dominio difficili da prevedere. 
Questa strategia ha un duplice vantaggio: da un lato, permette agli 
attaccanti di registrare solo alcuni dei domini generati, riducendo 
i costi e la visibilità; dall'altro, rende più difficile per le forze
dell'ordine e i ricercatori di sicurezza bloccare l'infrastruttura C2.

\subsubsection*{Conficker}
Uno dei primi (e più noti) esempi di malware che ha impiegato un 
algoritmo di generazione dei domini è conficker. Scoperto 
nel 2008, il suo DGA generava ogni giorno circa 250 domini diversi,
basandosi su un seed condiviso tra tutte le istanze del malware. 
Questo approccio permetteva al botmaster di registrare, secondo necessità,
uno di quei domini, e fornire istruzioni ai dispositivi infetti. Successive 
varianti di Conficker hanno ampliato il numero di domini generati 
fino a 50.000 al giorno, rendendo più difficile per i sistemi di sicurezza 
la loro identificazione e il loro blocco. 
\\\\
Il principale limite di Conficker era che il suo algoritmo di
generazione era relativamente semplice, e una volta identificato 
il metodo di creazione, era possibile prevedere i domini futuri e 
bloccarli preventivamente.

\subsubsection*{Gameover Zeus}
Gameover Zeus è stato un altro malware significativo che ha 
utilizzato DGA per eludere i sistemi di sicurezza. A differenza di 
Conficker, che generava i domini in modo relativamente semplice, 
Gameover Zeus impiegava un DGA più sofisticato che combinava fonti 
di entropia esterne, come i titoli di importanti testate giornalistiche, per 
generare domini difficili da prevedere. Questo tipo di tecniche rende complessa la 
creazione di blacklist efficaci, poiché i domini variavano in base a 
eventi non controllabili.

\subsubsection*{Necurs}
Necurs è una botnet che ha utilizzato un DGA avanzato per 
evitare il rilevamento. Questo malware ha introdotto un metodo di 
generazione basato su un modello statistico, producendo nomi di 
dominio apparentemente naturali, e quindi più difficili da individuare 
con filtri euristici (e.g. basati sull'entropia del nome di dominio). 
\\\\
Questa botnet è stata una delle più longeve e utilizzate per il malware 
finanziario, facilitando la diffusione di ransomware come Locky e 
banking trojan come Dridex.

\subsubsection*{Matsnu e suppobox}
Questi due malware sono esempi di DGA basati su dizionario. 
A differenza degli algoritmi pseudo-casuali che generano stringhe 
alfanumeriche casuali, Matsnu e Suppobox utilizzano 
parole prese da un dizionario per produrre domini apparentemente autentici. 
Questo approccio riduce la probabilità che il traffico 
verso questi domini venga classificato come sospetto, poiché i nomi 
generati appaiono simili a quelli utilizzati per siti legittimi.
\\\\
Uno degli elementi chiave che ha reso questi algoritmi così 
efficaci nel campo del cybercrimine è la loro capacità di 
adattarsi ai cambiamenti nei sistemi di difesa. 
Quando una tecnica viene identificata e bloccata, gli sviluppatori 
di malware affinano i loro DGA per renderli più sofisticati e 
resistenti alle contromisure.